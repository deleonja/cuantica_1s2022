\documentclass[11pt,letterpaper]{article}
\usepackage[utf8]{inputenc}
\usepackage[spanish]{babel}
\usepackage{amsmath}
\usepackage{amsfonts}
\usepackage{amssymb}
\usepackage{xcolor}
\usepackage{makeidx}
\usepackage{graphicx}
\usepackage{tikz}

\usepackage{physics}
\usepackage{dsfont}

\renewcommand{\theenumi}{\alph{enumi}}

\usepackage[draft,inline,nomargin]{fixme} \fxsetup{theme=color}
\definecolor{jacolor}{RGB}{200,40,0} 
\FXRegisterAuthor{ja}{aja}{\color{jacolor}JA}

\usepackage[left=2cm,right=2cm,top=2cm,bottom=2cm]{geometry}
\author{José Alfredo de León}
\title{Hoja de trabajo 2 \hspace{\fill}}
\begin{document}
%\maketitle
\noindent
\begin{tikzpicture}[x=1mm,y=1mm,overlay,remember picture]
  \pgftransformshift{\pgfpointanchor{current page}{center}}
  \node[inner sep=0pt] (usac) at (-76,108) %
  {\includegraphics[height=25mm]{ecfmByN}};
\end{tikzpicture}
%\includegraphics[width=0.2\textwidth]{logo1}
\hspace{23mm}
\begin{tabular}{p{149mm}}
Universidad de San Carlos de Guatemala \hspace*{\fill} Semestre 1, 2022 \\
Escuela de Ciencias Físicas y Matemáticas  \\
Mecánica Cuántica 1            \\
Profesor: Rodolfo Samayoa      \\
Auxiliar: José Alfredo de León \\
\end{tabular}

\begin{center}
\bf\Large Hoja de trabajo 3
\end{center}

\noindent
\textcolor{jacolor}{
Para el taller: 
\begin{itemize}
\item Revisar 
\end{itemize}
}

\noindent\textbf{Indicaciones importantes:}
\begin{itemize}
\item Utilice su lenguaje de preferencia para realizar esta hoja de trabajo.
Se sugiere utilizar Python o Mathematica.
\item Google es su mejor aliado para navegar en la programación. 
\item Existen librerías como Numpy para python que ya tienen implementadas
muchas rutinas, úselas. De igual forma, Mathematica tiene muchas funciones.
\end{itemize}

\subsection*{Ejercicio 1. }
Diseñe una rutina para generar una matriz cuadrada de dimensión $N\times N$
con entradas complejas y aleatorias.

\subsection*{Ejercicio 2. }
\janote{Esto ya existe en Mathematica}
Diseñe una rutina que revise la Hermiticidad de una matriz. La rutina
deberá devolver el valor booleano \verb|True| o \verb|False|.

\subsection*{Ejercicio 3.}
Utilizando la rutina del ejercicio anterior genere 100 matrices de $10\times10$
con entradas complejas y aleatorias y revise la Hermiticidad de dichas
matrices.

\subsection*{Ejercicio 4.}
Diseñe una rutina para calcular la probabilidad de medir a un sistema
en un estado 
$\ket{\psi}=\mqty(\alpha_1&\dots&\alpha_n)^T$
en el estado 
$\ket{\phi}=\mqty(\beta_1&\dots&\beta_n)^T.$
Esta rutina la utilizaremos luego dándole como entrada los eigenvectores que calculan las
rutinas \verb|linalg.eig| del módulo Numpy o \verb|Eigenvectors| de Mathematica, 
por lo que si está usando Python su rutina deberá asumir que $\ket{\psi}$
es un vector no normalizado, y si está usando Mathematica deberá 
asumir que tanto $\ket{\psi}$ como $\ket{\phi}$ son vectores no normalizados.

Diseñe otra rutina para calcular la misma probabilidad, pero considerando
que los eigenvalores pueden tener degeneración $d$, es decir, que $d$
vectores $\ket{\phi_{j}}$
\begin{align*}
P=\sum_{i=1}^d\frac{\abs{\braket{\phi_i}{\psi}}^2}{\braket{\phi_i}{\phi_i}\braket{\psi}{\psi}}
\end{align*}
Es decir, queremos asumir que $\ket{\psi}$ y $\ket{\phi}$ son vectores
no normalizados. Esto será útil en lenguajes como Mathematica en el cual
la rutina \verb|Eigenvectors| devuelve los eigenvectores no normalizados
de una matriz $M$.

Para esto puede diseñar dos rutinas diferentes o puede utilizar el 
\textit{polimorfismo} de los lenguajes de programación y diseñar 
dos rutinas con el mismo nombre, pero con diferentes entradas.

\subsection*{Ejercicio 5.}
Rutina para calcular el proyector dado un conjunto de vectores. 
Asumir que los vectores no están normalizados.

\subsection*{Ejercicio 6.}
Rutina para calcular valores de expectación dado un vector y un operador.

\subsection*{Ejercicio }
Con la rutina anterior diseñar otra rutina que implemente la incerteza.
\janote{puede ser}

\subsection*{Reto (x puntos extras netos)}
Rutina para encontrar si un conjunto finito de observables forman un CSCO.

\janote{También meter algo para hacer integrales.}



\end{document}