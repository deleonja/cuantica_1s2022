\documentclass[11pt,letterpaper]{article}
\usepackage[utf8]{inputenc}
\usepackage[spanish]{babel}
\usepackage{amsmath}
\usepackage{amsfonts}
\usepackage{amssymb}
\usepackage{xcolor}
\usepackage{makeidx}
\usepackage{graphicx}
\usepackage{tikz}

\usepackage{physics}
\usepackage{dsfont}

\renewcommand{\theenumi}{\alph{enumi}}

\usepackage[draft,inline,nomargin]{fixme} \fxsetup{theme=color}
\definecolor{jacolor}{RGB}{200,40,0} 
\FXRegisterAuthor{ja}{aja}{\color{jacolor}JA}

\usepackage[left=2cm,right=2cm,top=2cm,bottom=2cm]{geometry}
\author{José Alfredo de León}
\title{Hoja de trabajo 2 \hspace{\fill}}
\begin{document}
%\maketitle
\noindent
\begin{tikzpicture}[x=1mm,y=1mm,overlay,remember picture]
  \pgftransformshift{\pgfpointanchor{current page}{center}}
  \node[inner sep=0pt] (usac) at (-76,108) %
  {\includegraphics[height=25mm]{ecfmByN}};
\end{tikzpicture}
%\includegraphics[width=0.2\textwidth]{logo1}
\hspace{23mm}
\begin{tabular}{p{149mm}}
Universidad de San Carlos de Guatemala \hspace*{\fill} Semestre 1, 2022 \\
Escuela de Ciencias Físicas y Matemáticas  \\
Mecánica Cuántica 1            \\
Profesor: Rodolfo Samayoa      \\
Auxiliar: José Alfredo de León \\
\end{tabular}

\begin{center}
\bf\Large Hoja de trabajo 3
\end{center}

\noindent
\textcolor{jacolor}{
Para el taller: 
\begin{itemize}
\item Revisar 
\end{itemize}
}

\noindent\textbf{Indicaciones importantes:}
\begin{itemize}
\item No se calificará su procedimiento. Su calificación será proporcional
a la cantidad de incisos completos que resuelva. Por esto, 
deberá identificar bien (con color distinto, por ejemplo) el inciso que 
esté resolviendo.
\end{itemize}

\subsubsection*{Ejercicio 1. Pozo de potencial infinito.}
Una partícula de masa $m$ está sujeta al potencial 
\begin{align}
V(x)= \left\{ \begin{array}{lc}
             0, &   \abs{x}<a/2 \\
             \infty, &  \abs{x}>a/2. 
             \end{array}
   \right.
\end{align}
\begin{enumerate}
\item Encuentre las funciones de onda del estado fundamental, del primer 
y del segundo estado excitado.
\item Encuentre las expresiones para $E_1$, $E_2$ y $E_3$.
\item Encuentre los valores esperados 
$\expval{x}$ y $\expval{p}$ para el primer y segundo 
estado excitado.
\item Evalúe $\Delta x\Delta p$ para el primer y segundo estado 
excitado.
\end{enumerate}

\subsection*{Ejercicio 2. Un electrón a través de una barrera de potencial.}
Considere el potencial 
\begin{align}
V(x)= \left\{ \begin{array}{lc}
             6\text{ eV}, &   x<0 \\
             0, &  x>0. 
             \end{array}
   \right.
\end{align}
\begin{enumerate}
\item Un electrón con energía de $8$ eV se mueve de izquierda a derecha
en este potencial. Calcule la probabilidad de que el electrón i. continúe 
moviéndose en la dirección inicial al pasar la grada de potencial y 
ii. de ser reflejada por la grada de potencial.
\item Ahora suponga que el electrón se mueve de derecha a izquierda con 
una energía de 3 eV. i. Estime el orden de magnitud de la distancia que 
el electrón penetraría la barrera. ii. Repita la parte i. para una persona
de 70 kg que se mueve inicialmente a 4 m$/$s y corre hacia la pared que 
puede representarse por una grada de potencial de altura igual a cuatro veces
la energía de este individuo antes de aproximarse a la grada. 
\end{enumerate}

\subsection*{Ejercicio 3. Una partícula en un potencial infinito.}
Considere una partícula de masa $m$ que se mueve en un pozo de potencial infinito
unidimensional con paredes en $x=0$ y $x=a$, que se encuentra inicialmente en el 
estado 
\begin{align}
\psi(x,0)=\frac{1}{\sqrt{2}}\big[\phi_1(x)+\phi_3(x) \big],
\end{align}
donde $\phi_1(x)$ y $\phi_3(x)$ son el estado fundamental y el segundo estado 
excitado, respectivamente. 
\begin{enumerate}
\item Escriba la función de onda para el estado de la partícula para un tiempo 
$t>0$. 
\item Encuentre los valores esperados de $x$, $p$, $x^2$ y $p^2$.
\item Evalúe $\Delta x\Delta p$ y verifique que satisface el principio de incertidumbre.
\end{enumerate}

\subsection*{Ejercicio 4. Pozo de potencial finito.}
Considere una partícula de masa $m$ y energía $E$ sujeta al potencial 
\begin{align}
V(x)= \left\{ \begin{array}{rl}
             0, &   \quad x<a \\
             -V_0, &  \quad -a<x<a\\
             0,	   &  \quad x>a
             \end{array}
   \right.
\end{align}
Calcule los coeficientes de reflexión y transmisión $R$ y $T$ y evalúe 
qué ocurre cuando a. $E\gg V_0$, y b. $E\to 0$.

\subsection*{Ejercicio 5. Un electrón en una caja.}
Un electrón se mueve libremente adentro de un potencial infinito unidimensional 
con paredes en $x=0$ y $x=a$. Si el electrón se encuentra inicialmente en el 
estado fundamental ($n=1$) de la caja y súbitamente se cuadruplica el tamaño
de la caja (es decir, la pared derecha de la caja se mueve de $x=a$ a $x=4a$)
calcule la probabilidad de encontrar al electrón en
\begin{enumerate}
\item El estado fundamental de la nueva caja. 
\item El primer estado excitado de la nueva caja. 
\end{enumerate}


\end{document}